\documentclass[a4paper,twoside,onecolumn,openany,article]{memoir}
\usepackage{xeCJK}
\usepackage{url}
\usepackage{hyperref}
\usepackage{amsmath}
\usepackage{amssymb}
\usepackage{amsthm}
\usepackage{algpseudocode}
\usepackage{algorithm}
\defaultfontfeatures{Ligatures=TeX}

\setCJKmainfont[BoldFont=IPAGothic]{IPAMincho}
\setCJKsansfont{IPAGothic}
\setCJKmonofont{IPAGothic}



%\setmainfont{}
\setsansfont{URW Gothic}
\setmonofont{Inconsolata}

\usepackage{listings}



\settrimmedsize{\stockheight}{\stockwidth}{*}

%\setlrmarginsandblock{1.5in}{1in}{*}
\setlrmarginsandblock{1.5in}{1.5in}{*}
\setulmarginsandblock{1.3in}{1.5in}{*}
\setheadfoot{20mm}{15mm}

%\newlength{\linespace}
%\setlength{\linespace}{\baselineskip}
%\setlength{\headheight}{\onelineskip}
%\setlength{\headsep}{\linespace}
%\addtolength{\headsep}{-\topskip}

%\setlength{\footskip}{\onelineskip}
%\setlength{\footnotesep}{\onelineskip}

\checkandfixthelayout

\counterwithout{section}{chapter}
\setsecnumdepth{subsubsection}

\title{計算機科学第一}
\date{11月24日}

\begin{document}
\maketitle

\noindent
今週の目標
\begin{itemize}
\item 状態を使ったプログラムを書く
\end{itemize}

%\section{今日の課題}
\noindent
今日の課題(提出締切は...)
\begin{enumerate}
\item 
ScalaFX を使ったマンデルブロ集合のプログラムに「戻る」「進む」機能を持たせる。
\end{enumerate}

\noindent
今日のワークフロー
\begin{enumerate}
\item GitHub上の \verb|titech-is-cs115/assignment5| にあるプログラムを眺めて、仕組みを理解する。
\end{enumerate}

\section{課題}
変数と代入を使って状態を表現、更新することができます。
ScalaFX を用いて書かれたマンデルブロ集合のプログラムを実行してください。
このプログラムではマンデルブロ集合を表示しマウスでドラッグした範囲をズームして表示します。
このズームの操作により状態が変化していると考えられます。
このマンデルブロ集合のプログラムを読んでどこで状態が使われているか確認してください。
また状態の履歴を持たせることで「戻る」、「進む」機能を実装してください。

\end{document}
