\documentclass [11pt] {jsarticle}

\title {計算機科学第一最終レポート}
\author {高野成章}
\date {2016年2月15日提出}

\begin {document}
\maketitle

\section {はじめに}
今回の課題はお絵かきプログラムの実装である。大きく分けてバグの発見・修正および、新たな機能の追加を行った。複数の形が変えるようにオブジェクトを作成します。
また、その形に対して変換を及ぼす。得に色、削除、拡大が課題でありました。ここらはmandelbrotsetお同様に分離して考えることができると思います。指定された基本機能に加え、折れ線、打点、クリア(reset)を追加しました。
キーボードによる入力で制御したいという試みでCuiControlを設けましたが、実装には至らなかったので、実際のコードのほうではコメントアウトの状態にしています。

\section {図形オブジェクト}
\subsection{Ellipse}
rectangleを参考にしました。ellipseの場合は中心の座標、短径、長径を指定します。描画するとき、dragすることで中心および径が変わります。
中心はminを指定しているので、右下方向にdragしたときは最初にクリックした点、左上方向の場合はMouseEventが示す点になります。
\begin{verbatim}
object EllipseControl {

  var e  = new Ellipse {}
  var p0 = new Point2D(0,0)

  def onPress(ev: MouseEvent) {
   p0 = new Point2D(ev.x, ev.y)
   e = new Ellipse{
    centerX = p0.x ; centerY = p0.y
    stroke = strokeColor; fill = Color.Transparent
   }
   drawingPane.children += e
   shapes += TDEllipse(e)
    }

  def onDrag(ev: MouseEvent) {
    e.centerX = min(p0.x, ev.x);  e.centerY = min(p0.y, ev.y)
    e.radiusX = abs(p0.x-ev.x);   e.radiusY = abs(p0.y-ev.y)
  }

  def onRelease(ev: MouseEvent) {
    e.fill = fillColor
  }
}
\end{verbatim}

\subsection{Line}
線分の実装は始点(startX,startY)と終点(endX.endY)を指定することで求めています。
onDragで終点をマウスの位置に移動しています。\\
\begin{verbatim}
object LineControl {
 var l = new Line {}
 var p0 = new Point2D(0,0)

 def onPress(ev:MouseEvent){
 p0 = new Point2D(ev.x, ev.y)

  l = new Line \{
  startX = p0.x ; startY = p0.y
  endX = p0.x ; endY = p0.y
  stroke = strokeColor; fill = Color.Transparent
  }
  drawingPane.children += l
  shapes += TDLine(l)
  }

def onDrag(ev: MouseEvent){
 l.endX = ev.x ; l.endY = ev.y
 }

def onRelease(ev: MouseEvent){
 l.fill = fillColor
 }
}
\end{verbatim}

なお、線分の選択の実装のために、sublineも定義しました。sublineはlineを書くと同時に太い透明な線を後ろに書くことで、実際に選択しやすくなるというものです。
subline は変数をs(透明な線分)およびl(実際に引いている線分)を引数にとり、始点、終点を一致させることになっている。 \\

\begin{verbatim}
def subline(s:Line, l:Line){
   s.startX = l.startX() /*色を指定していないので透明になる*/
   s.startX = l.startY()
   s.endX = l.endX()
   s.endY = l.endY()
   s.strokeWidth = 20
}
\end{verbatim}

\subsection{Polyline}
Polyline(折れ線)も実装しました。
\begin{verbatim}
object PolylineControl {
  var p = new Polyline{strokeWidth = 2 }
  var p0 = new Point2D(0,0)
  var touch = 0

  def onPress(ev: MouseEvent){
   if (touch == 0){
    drawingPane.children += p
    shapes += TDPolyline(p)
    touch = 1
    }

   if (ev.clickCount != 1) {
    p = new Polyline{strokeWidth = 2}
    touch = 0
    }

    else {
    p0 = new Point2D(ev.x,ev.y)
    p.points ++= List(p0.x,p0.y)
    p.stroke = strokeColor
    }
 }
  def onRelease(ev: MouseEvent){
    p.stroke = strokeColor
  }
 }
\end{verbatim}

touch=0は制御点を打ったかどうかを調べるものです。touch = 0 の状態でクリックした場所が始点となります。
clickcountが1でない場合、
つまりダブルクリックをした場合などは、check を0に戻すことで、新たな始点の設定が可能になります。
check が 1の場合はpolyline のリストにクリックした点を格納し、点と点をつなぎます。
ここで問題点はpolylineの選択方法が実装できなかったため、
onDragを定義することや色を変えることもできません。

なお、Toggleのほうは次のように設定しました。

\begin{verbatim}
  new ToggleButton{
      id = "Polyline"
      graphic = new Polyline{
        stroke = Color.Black
         points ++= List(0,0,11,29,22,0,33,29)
      }
      toggleGroup = shapeGroup
    }
\end{verbatim}

List の値は(x0,y0,x1,y1,...)のように格納されています。ここに指定した値は、
選択ツールボックスの大きさが均一の大きさになるようにしています。
(選択ツールボックスのデザインが折れ線であり、その形を指定している)

\subsection{Dots}
打点の実装を行いました。打点といっても実際は楕円をdragするとともに書いていくというものです。
\begin{verbatim}
object DotsControl {
  var e = new Ellipse{}

  def onPress(ev:MouseEvent){
    e = new Ellipse{
    fill=strokeColor;stroke=strokeColor;e.radiusX=width
    e.radiusY=width;e.centerX=ev.x;e.centerY=ev.y
    }
    drawingPane.children += e
  }
  def onDrag(ev:MouseEvent){
    e = new Ellipse{
      fill=strokeColor;stroke=strokeColor;e.radiusX=width;e.radiusY=width
                      e.centerX=ev.x;e.centerY=ev.y
    }
    drawingPane.children += e
  }
}
\end{verbatim}

\section {コントロール部分}
\subsection{SelectControl}
\subsubsection{選択の判定}
線分の判定としては、sublineをcontains を用いて行いました。
ただし、この実装の問題点は実線lを中線としてsubline sができるのではなく、
sublineの一方の縁にlがくるようになっているというところです。
すなわち、選択できる領域は増えますが、lの上側あるいは下側についてのみ増えていることになります。
選択を行うことで影がつきますが、これを最新の選択のみにつける方法として、
選択した瞬間にすべてのケース(選択した場所が図形でないケースも)で影をnullにし、
そのあと選択したものだけに影をつけるようにした。
また、前にあるものから選択できるように、shapeを含むbufferにreverseをかましました。

\subsubsection{onDrag}
選択した図形を移動させるメソッドです。
長方形の移動に関して、(x1,y1)は最初の位置、(x0,y0)は図形をクリックした座標を示します。
ベクトルで考えると、
(r.x,r.y) – (x1,x0) = (ev.x, ev.y) – (x0,y0) が成り立つので、(r.x, r.y)が求まります。
線に関しては選択すると同時にp0を線分の中点にし、それを以下のようにして動かしています。
楕円は中心を動かしています。

\begin{verbatim}
def onDrag(ev: MouseEvent){
val x0 = p0.x ; val y0 = p0.y

selection match {
  case TDRectangle(r) => r.x =x1 + ev.x - x0 ; r.y = y1 + ev.y - y0
  case TDLine(l) => l.startX() = ev.x-p0.x ; l.startY() = ev.y -p0.y
                    l.endX = ev.x + p0.x; l.endY = ev.y + p0.y
  case TDEllipse(e) => e.centerX = ev.x; e.centerY = ev.y
  case TDNoShape() =>
 }
}
\end{verbatim}

\subsubsection{keyeventの実装}
keyeventの実装。
選択ボタン(Toggle)をクリック、図形をクリック、
そのあとdeleteボタンを押すことで選択した図形を消せるようにしました。
また、CTRL+rですべての図形を消去できるように設定ました。

\begin{verbatim}
def remove(s:TDShape):Unit = {
  s match {
    case TDRectangle(r) => drawingPane.children -= r
    case TDLine(l) => drawingPane.children -= l
    case TDEllipse(e) => drawingPane.children -=e
    case TDPolyline(p) => drawingPane.children -=p
  }
  shapes -= s
}

def reset = {
  drawingPane.children = Nil
  shapes = Buffer()
}
\end{verbatim}

\subsection{SizeControl}
図形の拡大縮小のところです。本当は選択の部分(SelectControl)に含めたかったのですが、
クリックした店の保存・固定および、onDragの部分を使い分ける必要があるので分離しました。
2つ目のselection matchでx1,y1(x2,y2)に形の長さ・大きさを保存し、
onDragのselection match で長さ、太さをMouseEventの座標の変化に応じて拡大縮小を行います。
例えば、長方形の場合-p0.xなる項がありますが、ここはSelectControlのonDragと同様に、
図形をクリックした地点を中心とり、そこからMouseEventがどれくらい動いたかを求める式です。

\begin{verbatim}
object SizeControl{
  var selection: TDShape = TDNoShape()
  var p0 = new Point2D(0,0)

  var x1:Double = 0 ; var y1:Double = 0
  var x2:Double = 0 ; var y2:Double = 0

  def onPress(ev:MouseEvent){
    val x = ev.x; val y = ev.y
    val oShape = shapes.reverse.find((shape: TDShape)=>
        shape match {
          case TDRectangle(r) => r.contains(x,y)
          case TDEllipse(e) => e.contains(x,y)
          case TDLine(l) => l.contains(x,y)
          case _ => false
    })

    selection = oShape match{
      case Some(shape) => shape
      case _ => TDNoShape()
    }

    selection match {
 		 case TDRectangle(r) => x1 = r.width() ; y1 = r.height()
 		 case TDEllipse(e) => x1 = e.radiusX(); x2 = e.radiusY()
 		 case TDLine(l) => x1 = l.startX(); y1 = l.startY()
                      x2 = l.endX(); y2 = l.endY()
     case _ =>
    }
    p0 = new Point2D(ev.x, ev.y)
  }

  def onDrag(ev:MouseEvent){
    val x = ev.x ; val y = ev.y
    selection match {
      case TDRectangle(r) => r.width = abs(x1 + ev.x-p0.x)
                             r.height = abs(y1+ev.y-p0.y)
      case TDEllipse(e) => e.radiusX = abs(x1+ev.x-p0.x)
                           e.radiusY = abs(y1+ev.y-p0.y)
      case TDLine(l) => l.startX = x1+ev.x-p0.x; l.startY= y1+ev.y-p0.y
                        l.endX=x2+ev.x-p0.x;l.endY=y2+ev.y-p0.y
    }
  }
}
\end{verbatim}

\subsection{CuiControl}
キーボードだけで操作を可能にする(すなわちToggleを推す必要がない)オブジェクトを設けましたが、
どういうわけか機能しませんでした。

\begin{verbatim}
object CuiControl{
  def onKeyPress(ev: KeyEvent){
    if (ev.isControlDown){
      ev.code match{
        case KeyCode.R => reset
        case KeyCode.ESCAPE => System.exit(0)
        case _ =>
        }
      }
    }
  }
\end{verbatim}


\section{Colorpickerおよびその他Toggleについて}
縁の色および、塗りつぶしに関しては、Colorpickerで選択した色をval c に保存し、
選択した図形によってstrokeColor、fillColorを行うmatch case 文を書きました。
\begin{verbatim}
val colorTools = Seq(
  new ColorPicker(strokeColor) {
    onAction = { e: ActionEvent => strokeColor = value()
    val c = value()
    strokeColor = Color.hsb(c.hue, c.saturation, c.brightness, 0.5)
    SelectControl.selection match {
        case TDRectangle(r) => r.stroke = strokeColor
        case TDEllipse(e) => e.stroke = strokeColor
        case TDLine(l) => l.stroke = strokeColor
        case TDPolyline(p) => p.stroke = strokeColor
        case _ => ()
        }
    }
  },
  new ColorPicker(fillColor) {
    onAction = { e: ActionEvent =>
      val c = value()
      fillColor = Color.hsb(c.hue, c.saturation, c.brightness, 0.5)
      SelectControl.selection match {
        case TDRectangle(r) => r.fill = fillColor
        case TDEllipse(e) => e.fill = fillColor
        case TDLine(l) => l.fill = fillColor
        case _ => ()
      }
    }
  })
\end{verbatim}
\section{その他(時間の制約上)修正できなかった点}
\begin{itemize}
\item 図形がdrawingPaneを超えるサイズになってしまう。選択ツールにオーバーラップすると終了するしかない。
\item polyline,Dotsを書いてからは、選択機能が使えなくなる。
それらを選択せざるを得ないのだが、選択のメソッドがオブジェクトに定義されていないのが問題点
\item traitを用いてSelectControl,SizeControlやEllipseControl,RectangleControl
を共通化すること
\end{itemize}
\section {おわりに}
実装するにあたって、他の人に助言をいただいたり、scalafxのサンプルコードなども参考にしました。
その上でコードを理解し、使いこなせるような訓練ができたと思います。
一からオブジェクトをデザインするにはもっとたくさんの経験が必要だと感じます。
まずは読解力を高め、新たに創造する技術は今後の課題します。

前期はカリー化などをはじめ、基本的なメソッドを一から作る練習をしましたが、
後期では既存のライブラリを使いこなす能力が必要でしたが、
開発には両方の能力が備わっていることが望ましいのだと感じます。
scalafxのドキュメントですが、英語の問題ではなく、例の説明が少ないため使いこなすことができませんでした。
scalafxに関するテストについてですが、GUIプログラミング寄りに感じたので、
ユーザーが報告する(すなわち、実際に動かして試してみる)というのが効率的だと感じました
(少なくとも短期間の間での実装で
)。
\end {document}
